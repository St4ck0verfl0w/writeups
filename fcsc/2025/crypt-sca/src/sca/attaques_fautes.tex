\begin{frame}{Attaques par faute sur RSA et ECDSA}
    \large{\centerline{\textbf{Bellcore et LLL}}}
\end{frame}

\begin{frame}{No Divide just Conquer \FiveStar/\FiveStar\FiveStar/\FiveStar\FiveStar\FiveStar \hfill 60/\textcolor{red}{21/6 résolutions}}
    \begin{columns}[c]
        \column{.50\textwidth}
        \begin{center}                  
            \includegraphics[width=0.9\textwidth]{img/meme/rsa-intro.png}
        \end{center}

        \column{.50\textwidth} % 
           \begin{outline}
               \1 Objectif
                \2 Implémenter RSA en assembleur-like
                \2 Avec de plus en plus de restrictions
           \end{outline}
    \end{columns}
\end{frame}

\begin{frame}{Atomic Secable \FiveStar\FiveStar\FiveStar \hfill \textcolor{red}{4 résolutions}}
\begin{columns}[c]
        \column{.50\textwidth}
        \begin{center}                  
            \includegraphics[width=0.65\textwidth]{img/meme/atomic-secable-intro.png}
        \end{center}

        \column{.50\textwidth} %
           \begin{outline}
                \1 Objectif
                    \2 Récupérer la clef secrète ECDSA
                \pause
                \1 Données
                    \2 65 536 signatures fautées avec la clef
                \pause
                \1 Capacités d'attaquant
                    \2 Fauter aléatoirement une liste d'instruction donnée
           \end{outline}
    \end{columns}
\end{frame}

\begin{frame}{Attaque à information partielle sur ECDSA \footnote{\cite{demicheli:hal-03045663}}}

\[\left\{
\begin{array}{c c c}
\k{s_1} & = &\u{k_1}^{-1}\left(\k{h_1}+\k{r_1}*\u{d}\right) \mod \k{n} \\
\k{s_2} & = &\u{k_2}^{-1}\left(\k{h_2}+\k{r_2}*\u{d}\right) \mod \k{n} \\
&\vdots& \\
\k{s_m} & = &\u{k_m}^{-1}\left(\k{h_m}+\k{r_m}*\u{d}\right) \mod \k{n} \\
\end{array}
\right.
\pause
\;\Rightarrow\;
\left\{
\begin{array}{c c c}
\u{k_1}+\k{t_1}\u{k_m} + \k{u_1} & = & 0 \mod \k{n} \\
\u{k_2}+\k{t_2}\u{k_m} + \k{u_2} & = & 0 \mod \k{n} \\
&\vdots&\\
\u{k_{m-1}}+\k{t_{m-1}}\u{k_m} + \k{u_{m-1}} & = & 0 \mod \k{n} \\
\end{array}
\right.\]

\pause
\begin{center}
    On espère que les $\u{k_i}$ soient suffisamment petits (connaître les bits de poids fort)
    \pause
    
    Le réseau suivant contient le vecteur $(\u{k_1},\;\dots,\;\u{k_m},\;\k{K})$
\end{center}
    \begin{columns}[c]
        \column{.35\textwidth}
        \[\left(
        \begin{array}{c c c c c c}
        \k{n} &   &        &   &   &    \\
          & \k{n} &        &   &   &   \\
          &   & \ddots &   &   &    \\
          &   &        & \k{n} &   &   \\
        \k{t_1}  &   \k{t_2} &  \dots    &  \k{t_{m-1}}  & 1 &   \\
        \k{u_1}  &   \k{u_2} &  \dots    &  \k{u_{m-1}}  & 0 & K  \\
        \end{array}
        \right)\]
        \pause
        \column{.65\textwidth} % 
           \begin{outline}
           \1 Tricks sur l'algorithme LLL
            \pause
            \2 La diagonale de $\k{n}$ permet de simuler un réseau modulaire
            \pause
            \2 K est choisi grand pour forcer la dernière ligne à 1
           \end{outline}
    \end{columns}


\end{frame}


\begin{frame}{Dans le cadre du challenge}
    \begin{columns}[c]
\column{.55\textwidth}
        \begin{outline}
            \1Retrouver la clef privée: si un des $\u{k_i}$ \textbf{ou certains bits de plusieurs $\u{k_i}$} sont connus, c'est gagné
            
            \uncover<2->{
            \1 Objectif : distinguer les deux cas
            }
                \uncover<3->{ 
                \2 Par SCA (opérations différentes)
                }
                \uncover<4->{
                \2 Par timing (premières opérations ignorées)
                    \3 Temps d’exécution non constant
                }
                \uncover<5->{
                \2 Un seul des deux cas resiste à une faute
                    \3 On faute les 16 premières opérations
                    \3 Les survivants commencent par 16 zéros
                }
        \end{outline}
\column{.45\textwidth} % 
    \uncover<2->{
      \begin{algorithm}[H]
        \SetAlgoLined
        \KwIn{Scalaire $k = (k_{n-1}, \dots, k_0)_2$, Point $P$}
        \KwOut{$Q = [k]P$}
        $(R_0,R_1) \leftarrow (\mathcal{O},P)$\;
        \For{$i \leftarrow n-1$ \KwTo $0$}{
            \If{$k_i = 0$}{
                $(R_0, R_1) \leftarrow ([2]R_0,R_0 + R_1)$\;
            }
            \Else{
                $(R_0,R_1) \leftarrow (R_0 + R_1,[2]R_1)$\;
            }
        }
        \Return $R_0$\;
        \caption{Echelle de Montgomery}
    \end{algorithm}
    }
    \end{columns}

\end{frame}

